%
% LaTeX source of my resume
% =========================
%
% Heavily commented to to fit even LaTeX beginners (hopefully).
%
% See the `README.md` file for more info.
%
% This file is licensed under the CC-NC-ND Creative Commons license.
%


% Start a document with the here given default font size and paper size.
\documentclass[10pt,a4paper]{article}

% Set the page margins.
\usepackage[a4paper,margin=0.75in]{geometry}

% Setup the language.
\usepackage[english]{babel}
\hyphenation{Some-long-word}

% Makes resume-specific commands available.
\usepackage{resume}




\begin{document}  % begin the content of the document
\sloppy  % this to relax whitespacing in favour of straight margins


% title on top of the document
\maintitle{Jimmy Callin}{August 27, 1990}{Last update on \today}

\nobreakvspace{0.3em}  % add some page break averse vertical spacing

% \noindent prevents paragraph's first lines from indenting
% \mbox is used to obfuscate the email address
% \sbull is a spaced bullet
% \href well..
% \\ breaks the line into a new paragraph
\noindent\href{mailto:jimmy.callin@gmail.com}{jimmy.callin@gmail.com}\sbull
\textsmaller{+}46 70 540 45 14
\\
Luthagsesplanaden 87\sbull
752 71\thinspace {\large}\sbull
Uppsala\sbull
Sweden

\spacedhrule{0.9em}{-0.4em}  % a horizontal line with some vertical spacing before and after

\roottitle{Summary}  % a root section title

\vspace{-1.3em}  % some vertical spacing
\begin{multicols}{2}  % open a multicolumn environment
\noindent \emph{An ambitious and curious student who is currently finishing his Master's degree in Language Technology.}
\\
I have since picking up Chinese been having a great interest in language no matter what area. When the academic language studies concluded in Shanghai 2011, I naturally chose to continue within the area of computational linguistics, which has been the main topic of studies for these past four years.

I wrote my Bachelor's thesis on using compositions of distributional semantic vectors in sentiment classification. Next semester, for my Master's thesis, I intend to continue to work on distributed representations.

Please see my Github page for a portfolio of code projects: \href{https://github.com/jimmycallin}{github.com/jimmycallin}
\end{multicols}


\spacedhrule{0em}{-0.4em}

\roottitle{Experience}

\headedsection  % sets the header for the section and includes any subsections
  {\href{http://gavagai.se}{Gavagai AB}}
  {\textsc{Stockholm, Sweden}} {%
  \headedsubsection
    {Support Manager \& Editor}
    {Jun. 2012 -- Jun. 2015}
    {\bodytext{Gavagai is a Stockholm start-up specializing in big data text analytics, involving some of Sweden's foremost experts within the area. My day-to-day work 
    concerned being responsible for technical support, where I was an essential part in building a support process which assured that customers' technical inquiry were dealt with in a transparent and efficient manner.
    
I curated and constructed targets of interest in various languages and key domains for business intelligence use. I have also developed several internal tools for data analysis, with work in web development, general programming and various natural language processing areas such as automatic sentiment analysis evaluation, Chinese word segmentation, tokenization and transliteration.}}
}


\vspace{1em}

\spacedhrule{-0.2em}{-0.4em}

\roottitle{Publications}

\headedsection  % sets the header for the section and includes any subsections
  {\href{http://aclweb.org/anthology/W15-2508}{Part-of-Speech Driven Cross-Lingual Pronoun Prediction with Feed-Forward Neural Networks.}}
  {\textsc{}} {%
  \headedsubsection
    {Proceedings of the Second Workshop on Discourse in Machine Translation (DiscoMT), Stroudsburg, PA: Association for Computational Linguistics, 2015, 59-64}
    {September 2015}
    {\bodytext{For some language pairs, pronoun translation is a discourse-driven task which requires information that lies beyond its local context. This motivates the task of predicting the correct pronoun given a source sentence and a target translation, where the translated pronouns have been replaced with placeholders. For cross-lingual pronoun prediction, we suggest a neural network-based model using preceding nouns and determiners as features for suggesting antecedent candidates. Our model scores on par with similar models while having a simpler architecture.}}}


\vspace{1em}

\spacedhrule{-0.2em}{-0.4em}


\roottitle{Education}

\headedsection
  {\href{http://www.uu.se/en/}{Uppsala University}}
  {\textsc{Uppsala, Sweden}} {%
  
  \headedsubsection
    {Master's degree in Language Technology}
    {Aug. 2014 -- Jun. 2016}
    {\bodytext{Continuing my efforts in natural language processing, this time with a greater focus on research projects.}}
  
  \headedsubsection
    {Bachelor's degree in Language Technology}
    {Aug. 2011 -- Jun. 2014}
    {\bodytext{Focused on using applied computer science for modeling natural languages. Elective courses were concentrated on mathematics and statistics.}}
    
}

\headedsection
  {\href{http://gu.se/english/}{Gothenburg University, Zhejiang University, East China Normal University}}
  {\textsc{Sweden and China}} {%
  \headedsubsection
    {Chinese language and culture}
    {Aug. 2009 -- Jun. 2011} {}
}

\spacedhrule{0.5em}{-0.4em}

\roottitle{Skills}

\inlineheadsection  % special section that has an inline header with a 'hanging' paragraph
  {Technical expertise:}
  {Codes daily in Python and occasionally in Java. Experience with Javascript, Matlab, C(++), Ruby, and SML.}

\vspace{0.5em}
\inlineheadsection
  {Communication skills:}
  {Swedish \emph{(native language)}, English \emph{(full professional proficiency)}, \\Chinese \emph{(limited working proficiency)}.}

\end{document}

＀
